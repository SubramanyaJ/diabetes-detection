\documentclass{article}
\usepackage[english]{babel}

% Or, use letterpaper
\usepackage[a4paper ,top=2cm,bottom=2cm,left=3cm,right=3cm,marginparwidth=1.75cm]{geometry}

% Packages
\usepackage{amsmath}
\usepackage{graphicx}
\usepackage[colorlinks=true, allcolors=blue]{hyperref}

\title{paper-title}
\author{author-namelist}

\begin{document}
\maketitle

\begin{abstract}
%	This study uses machine learning to diagnose diabetes patients. Naive Bayes Classifier, K-Nearest Neighbours, Logistic Regression, Random Forest and XGBoost models were the algorithms used, with XGBoost showing the most promise. The XGBoost model was thus selected and a visual representation of the importance of each attribute was derived. 
%	Present models suffer from being unable to differentiate between different types of diabetes due to lack of data points in a single dataset. This model hopes to alleviate this weakness by compounding two datasets, such that one predicts the presence of diabetes, and the other, the type of diabetes.


\end{abstract}

\section{Introduction}


\section{Literature Review}


\subsection{Subsection if needed}

\section{Methodology}


\end{document}
